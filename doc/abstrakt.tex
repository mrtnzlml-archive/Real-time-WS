\chapter*{Abstrakt}
V této práci je hlavním úkolem vyřešit real-time komunikaci mezi senzorickou sítí a webovou aplikací. Toho je dosaženo použitím klasické komunikace TCP a UDP. Hlavní složkou celé práce je JavaScriptový server s databází Redis, který zajišťuje výměnu informací mezi koncovými členy celé sítě. Tyto členy jsou tvořeny mikrokontroléry od společnosti STMicroelectronics, které jsou k serveru připojeny pomocí běžných síťových prvků a Ethernetu. Výsledkem práce je funkční program, jehož funkčnost byla ověřena na praktickém zapojení sítě.

\vfill

\section*{Klíčová slova}
Ethernet, Sails.js, Node.js, Mikrokontrolér, Redis, RESP, TCP, UDP, Websocket, STMicroelectronics