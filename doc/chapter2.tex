\chapter{Volba vhodné technologie}
Pro tuto síť nejsou v tuto chvíli stanoveny zadávající firmou žádné konkrétní požadavky na hardware. Proto je možné vybrat z hlediska softwarového řešení jakoukoliv platformu. Z hardwarového hlediska je doporučeno používat evaluační desky od STMicroelectronics. V následující části budu popisovat jednotlivé použité technologie a důvod jejich volby.

\section{Prvky senzorické sítě}
%TODO popsat také druhou desku (40x)


\section{Real-time server}
%TODO Node.js


\section{Databázový server}
%TODO Redis
\cite{redis-benchmark}

\begin{minted}[linenos,breaklines]{text}
$ redis-benchmark -q -n 100000 -d 256
PING_INLINE: 212314.23 requests per second
PING_BULK: 211416.50 requests per second
SET: 131752.31 requests per second
GET: 199600.80 requests per second
INCR: 213219.61 requests per second
LPUSH: 213219.61 requests per second
LPOP: 204918.03 requests per second
SADD: 214592.28 requests per second
SPOP: 212765.95 requests per second
LPUSH (needed to benchmark LRANGE): 213675.22 requests per second
LRANGE_100 (first 100 elements): 45269.35 requests per second
LRANGE_300 (first 300 elements): 15586.04 requests per second
LRANGE_500 (first 450 elements): 9325.75 requests per second
LRANGE_600 (first 600 elements): 6472.49 requests per second
MSET (10 keys): 131578.95 requests per second
\end{minted}

\begin{minted}[linenos,breaklines]{text}
$ redis-benchmark -q -n 100000 -d 256 -P 16
PING_INLINE: 1612903.25 requests per second
PING_BULK: 2127659.75 requests per second
SET: 1086956.50 requests per second
GET: 1351351.38 requests per second
INCR: 1219512.12 requests per second
LPUSH: 934579.44 requests per second
LPOP: 1030927.81 requests per second
SADD: 1265822.75 requests per second
SPOP: 1562499.88 requests per second
LPUSH (needed to benchmark LRANGE): 990099.00 requests per second
LRANGE_100 (first 100 elements): 35186.49 requests per second
LRANGE_300 (first 300 elements): 8521.52 requests per second
LRANGE_500 (first 450 elements): 5236.70 requests per second
LRANGE_600 (first 600 elements): 3888.48 requests per second
MSET (10 keys): 207468.88 requests per second
\end{minted}

\subsection{RESP protokol}
Redis databáze komunikuje interně přes TCP v RESP (Redis Serialization Protocol) formátu. RESP používá celkem 5 typů dat. Vždy platí, že první byte je byte určující o jaký formát se jedná:

\begin{itemize}
\itemsep0em
\item \texttt{+} jednoduchý string
\item \texttt{-} error
\item \texttt{:} integer
\item \texttt{\$} bulk string (binary safe)
\item \texttt{*} array
\end{itemize}

Následuje samotný obsah, nebo dodatečné informace, například o délce a vše je ukončeno pomocí \texttt{CRLF} (\texttt{\textbackslash r\textbackslash n}). Postupně tedy přenášené informace mohou vypadat například takto:

\begin{itemize}
\itemsep0em
\item \texttt{+PONG\textbackslash r\textbackslash n}
\item \texttt{-Error 123\textbackslash r\textbackslash n}
\item \texttt{:54986\textbackslash r\textbackslash n}
\item \texttt{\$4\textbackslash r\textbackslash nPING\textbackslash r\textbackslash n} (první část určuje délku bulk stringu, NULL je pak \texttt{\$-\textbackslash r\textbackslash n})
\item \texttt{*2\textbackslash r\textbackslash n\$3\textbackslash r\textbackslash nGET\textbackslash r\textbackslash n\$3\textbackslash r\textbackslash nkey\textbackslash r\textbackslash n} (první je délka pole, následuje kombinace předchozích)
\end{itemize}

Redis server potom přijímá podle bulk stringů obsahující jednotlivé instrukce. Tento protokol je velmi důležitý, protože i koncentrátory posílají data (přes TCP i UDP) v RESP formátu, je tak možné data posílat přímo do databáze. Tato vlastnost však není využívána, protože je vhodné, aby byl jako prostředník server a například zjišťoval aktivitu koncentrátorů. Každopádně tato možnost zde je a pokud by bylo zapotřebí ukládat data tou nejrychlejší cestou, přímý přístup do databáze je tímto možný a funkční.

\section{Webová aplikace}
%TODO Sails.js, Express, Sockets.io