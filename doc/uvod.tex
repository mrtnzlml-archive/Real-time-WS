\chapter{Úvod}
\pagenumbering{arabic}
Cílem této práce je navrhnout komunikaci pro senzorickou síť s~přihlédnutím k~tomu, že by tato síť měla být ovladatelná v~reálném čase z~webové řídicí aplikace. Toto je velmi zásadní požadavek pro budoucí realizaci, protože z~hlediska elektronických systémů je real-time \index{Real-time} komunikaci možné realizovat pomocí protokolů k~tomu určených, které vymezují přenos dat do přesně definovaných časových slotů (Ethernet Powerlink, \index{Ethernet Powerlink} Time-triggered CAN, FlexRay). \index{FlexRay} U~webových aplikací žádný takový prvek neexistuje a webová řídicí aplikace se tak stává limitujícím prvkem celé sítě. Existují však metody, které se real-time komunikaci, resp. rychlé komunikaci, jak je real-time u~webových aplikací všeobecně chápán, mohou velmi přiblížit. V~roce 2011 bylo vydáno RFC 6455 \cite{rfc6455}, které zastřešuje nový protokol \index{Websocket} websocket, který umožňuje propojení serveru a klientské části aplikace socketem a je tak možné přenášet informace velmi vysokou rychlostí, což doposud nebylo prakticky téměř možné realizovat.

V~následující části práce bude rozebrána problematika komunikace senzorické sítě s~webovou řídicí aplikací, ze které vyplyne, že nejvhodnějším řešením je naprogramovat jednotlivé členy senzorické sítě co nejvíce ní\-zko\-ú\-rov\-ňo\-vě, následně je propojit s~řídicím serverem, na kterém poběží \index{Node.js} Node.js real-time server (asynchronní single-thread) pro zpracovávání požadavků a zároveň zde poběží server pro webovou aplikaci, která bude využívat websocket protokolu coby nástroje pro komunikaci s~tímto serverem. Zároveň je tato senzorická síť uváděna na příkladu administrativní budovy, resp. jakéhokoliv objektu, kde se běžně pohybují lidé a využívají konvenční elektroinstalaci, tzn. například domácí objekty, popřípadě jiné objekty podobného charakteru, kde má využití této sítě praktický přínos.