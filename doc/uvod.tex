\chapter{Úvod}
\pagenumbering{arabic}
Cílem této práce je navrhnout real-time komunikaci pro senzorickou síť s přihlédnutím k tomu, že by tato síť měla být ovladatelná z webové aplikace. Toto je velmi zásadní požadavek pro budoucí realizaci, protože z hlediska elektronických systémů je real-time komunikaci možné realizovat pomocí protokolů k tomu určených, které provádí časové korekce (Ethernet Powerlink, Time-triggered CAN, FlexRay). U webových aplikací žádný takový prvek neexistuje a webová řídící aplikace se tak stává limitujícím prvkem celé sítě. Existují však metody, které se real-time komunikaci mohou velmi přiblížit. V roce 2011 bylo vydáno \cite[RFC 6455]{rfc6455}, které zastřešuje nový protokol websocket, který umožňuje propojení serveru a klientské části aplikace socketem a je tak možné přenášet informace velmi vysokou rychlostí, což doposud nebylo prakticky téměř možné realizovat.

V následující části práce bude rozebrána problematika komunikace senzorické sítě s webovou řídící aplikací, ze které vyplyne, že nejvhodnějším řešením je naprogramovat jednotlivé členy senzorické sítě co nejvíce nízkoúrovňově, následně je propojit s řídícím serverem, na kterém poběží Node.js real-time server pro zpracovávání požadavků a zároveň zde poběží server pro webovou aplikaci, která bude využívat websocket protokolu coby nástroje pro komunikaci s tímto serverem. Zároveň je tato senzorická síť uváděna na příkladu rodinného domu, popřípadě jiných objektů, kde by byla jinak rozvedena klasická elektroinstalace.