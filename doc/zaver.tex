\chapter{Závěr}
Tato práce byla pro mě velkým přínosem zejména díky tomu, že jsem si mohl prakticky vyzkoušet JavaScriptový real-time server Node.js v kombinaci s key-value databází Redis. Vzhledem ke svým speciálním vlastnostem není moc aplikací, kde lze tyto technologie použít. Celý efekt je umocněn tím, že jsem  mohl pracovat s mikrokontroléry od STMicroelectronics. Spojení webové aplikace a elektroniky je v dnešní době stále velmi nezvyklé, ale pomalu nabírá na popularitě.

%TODO aktualizovat počet přenesených paketů

V práci se podařilo úspěšně vyřešit veškeré body zadání. Kromě teoretického rozboru a výběru vhodných technologií, byla celá myšlenka naprogramována a několik stovek hodin postupně testována. Celkem bylo vyměněno mezi koncentrátory a serverem více než $25\,000\,000$ paketů nesoucích data, což odpovídá zhruba $1,25\,GB$ přenesených testovacích dat při průměrné velikosti paketu $50\,B$. Při tomto testování se potvrdila původní myšlenka, tedy že ovládat síť pomocí přenášené informace má mnohem více možností, než ovládání toku energie v rozvodu a je možné něco takového prakticky zrealizovat.

Hlavním nedostatkem a překážkou na cestě k širšímu praktickému využití je nutnost vytvoření zcela nových ovládacích prvků v běžných elektrorozvodech. To se týká prakticky jakéhokoliv zařízení, protože tato myšlenka počítá s tím, že bude možné s libovolným prvkem sítě komunikovat a předávat si informace. Toto je však překážka, která bude vzhledem ke vzrůstající popularitě IoT brzy překonána. Tato síť byla v práci navržena a vyzkoušena jako metalická. Samotný návrh však není na vodiče nijak vázaný a lze využít v kombinaci s bezdrátovou komunikací například na principu Wi-Fi, nebo Li-Fi.