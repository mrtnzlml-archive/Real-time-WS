\documentclass[oneside,12pt,a4paper,final]{book} %opt: draft/final
\usepackage[czech]{babel}
\usepackage[utf8]{inputenc}
\usepackage{graphicx} % support the \includegraphics command and options
\usepackage{hyperref} % links in \tableofcontents
\usepackage{minted} % syntax highlighter
\usepackage{pdfpages}
\usepackage{textcomp}
\usepackage{float} % figure [H]
\usepackage{pgfplots} % tikzpicture
\usepackage{amsmath} % linebreaks in math mode
\usepackage{dirtree} % struktura slozek
\usepackage{makeidx} % rejstrik

\makeindex

\hoffset=0.5cm % posune stranku ->

\hypersetup{
	colorlinks,
	citecolor=black, %blue
	filecolor=black,
	linkcolor=black, %red
	urlcolor=black %blue
}
\addto\captionsczech{\renewcommand{\chaptername}{}} %remove "chapter" word

\pgfplotsset{major grid style={lightgray!50!white}}
\pgfplotsset{soldot/.style={color=blue,only marks,mark=*}}
\pgfplotsset{holdot/.style={color=blue,fill=white,only marks,mark=*}}

\makeatletter
\newcommand{\tocfill}{\cleaders\hbox{$\m@th \mkern\@dotsep mu . \mkern\@dotsep mu$}\hfill}
\makeatother
\newcommand{\abbrlabel}[1]{\makebox[6cm][l]{\textbf{#1}\ \tocfill}}
\newenvironment{abbreviations}{\begin{list}{}{\renewcommand{\makelabel}{\abbrlabel}%
        \setlength{\labelwidth}{6cm}\setlength{\leftmargin}{\labelwidth+\labelsep}%
                                              \setlength{\itemsep}{0pt}}}{\end{list}}

\newcommand{\namesig}[2][5cm]{%
  \begin{tabular}{@{}p{#1}@{}}
    #2 \\[5\normalbaselineskip] \dotfill \\[0pt]
    {\small \textit{Podpis}}
  \end{tabular}
}

\interfootnotelinepenalty=10000 %BRUTAL!

\begin{document}

%%% TITLE
\pagestyle{empty}
\begin{titlepage}
\noindent
\begin{center}
	{\LARGE ZÁPADOČESKÁ UNIVERZITA V~PLZNI} \\[0.1cm]
	{\LARGE FAKULTA ELEKTROTECHNICKÁ} \\[0.4cm]
	{\Large\sc Katedra elektroenergetiky a~ekologie} \\
	\vspace{5cm}
	{\Huge\sc Bakalářská práce} \\
	\vspace{1cm}
	{\large Návrh a realizace real-time komunikace pro senzorickou síť\\s webovou řídicí aplikací\\}
	\vspace{1cm}
	{\large Design and Implementation of Real-time Communication for Sensory Network with Website Based Control Application}
\end{center}
\vfill
Autor práce: Martin Zlámal\\
Vedoucí práce: Ing. Petr KRIST, Ph.D. \hfill Plzeň 2015
\end{titlepage}

\pagestyle{plain}
\includepdf[pages={1}]{img/zadani1.jpg}
\includepdf[pages={1}]{img/zadani2.jpg}
\chapter*{Abstrakt}
Text abstraktu v češtině...

\vfill
\section*{Klíčová slova}
Ethernet, Expres.js, Node.js, Procesor, Redis, RESP, TCP, UDP, Websocket
\pagenumbering{roman}
\chapter*{Abstract}
Text abstraktu v angličtině...

\vfill
\section*{Keywords}
\chapter*{Prohlášení}
Předkládám tímto k posouzení a obhajobě bakalářskou práci, zpracovanou na závěr studia na Fakultě elektrotechnické Západočeské univerzity v Plzni.

Prohlašuji, že jsem svou závěrečnou práci vypracoval samostatně pod vedením vedoucího diplomové práce a s použitím odborné literatury a dalších informačních zdrojů, které jsou všechny citovány v práci a uvedeny v seznamu literatury na konci práce. Jako autor uvedené diplomové práce dále prohlašuji, že v souvislosti s vytvořením této závěrečné práce jsem neporušil autorská práva třetích osob, zejména jsem nezasáhl nedovoleným způsobem do cizích autorských práv osobnostních a jsem si plně vědom následků po\-ru\-še\-ní ustanovení § 11 a následujících autorského zákona č. 121/2000 Sb., včetně možných trestněprávních důsledků vyplývajících z ustanovení § 270 trestního zákona č. 40/2009 Sb.
\chapter*{Poděkování}
Děkuji vedoucímu této bakalářské práce Ing. Petrovi Kristovi, Ph.D. za svůj čas věnovaný přípravě této práce při konzultacích a za motivaci, která mě nutila posouvat své hranice stále dál. Mé díky patří také společnosti UNIOSO s.r.o. za cenné připomínky při návrhu celého systému a za poskytnutí vývojových desek značky STMicroelecronics, bez kterých by vytvoření této práce nikdy nebylo možné.


\tableofcontents
\cleardoublepage
\phantomsection
\addcontentsline{toc}{chapter}{\listfigurename}
\listoffigures


\chapter*{Seznam symbolů a zkratek}
\addcontentsline{toc}{chapter}{Seznam symbolů a zkratek}
\noindent
\begin{abbreviations}
\item[ADC]		Analog-to-Digital Converter
\item[AJAX]		Asynchronous JavaScript and XML
\item[AJAJ]		Asynchronous JavaScript and JSON
\item[BGA]		Ball Grid Array
\item[BSP]		Board Support Package
\item[CAN]		Controller Area Network
\item[CPU]		Central Processing Unit
\item[CRLF]		Carriage Return Line Feed
\item[DHCP]		Dynamic Host Configuration Protocol
\item[EDA]		Event-Driven Architecture
\item[EJS]		Embedded JavaScript
\item[HAL]		Hardware Abstraction Layer
\item[HTTP]		Hypertext Transfer Protocol
\item[ICMP]		Internet Control Message Protocol
\item[I/O]		Input/Output
\item[IoT]		Internet of Things
\item[IP]		Internet Protocol
\item[JS]		JavaScript
\item[JSON]		JavaScript Object Notation
\item[JSONP]	JSON with padding
\item[LED]		Light-Emitting Diode
\item[MAC]		Media Access Control
\item[MVC]		Model View Controller
\item[NPM]		Node Package Manager
\item[OS]		Operating System
\item[PWM]		Pulse Width Modulation
\item[RAM]		Random-Access Memory
\item[RESP]		Redis Serialization Protocol
\item[RFC]		Request for Comments
\item[RTT]		Round Trip Time
\item[SMA]		Simple Moving Average
\item[TCP]		Transmission Control Protocol
\item[TED]		Technology, Entertainment, Design
\item[UDP]		User Datagram Protocol
\item[UTP]		Unshielded Twisted Pair
\item[UFBGA]	Ultra Fine BGA
\item[XHR]		XMLHttpRequest
\item[XML]		Extensible Markup Language
\end{abbreviations}



\chapter{Úvod}
\pagenumbering{arabic}
Cílem této práce je navrhnout komunikaci pro senzorickou síť s přihlédnutím k tomu, že by tato síť měla být ovladatelná v reálném čase z webové řídící aplikace. Toto je velmi zásadní požadavek pro budoucí realizaci, protože z hlediska elektronických systémů je real-time \index{Real-time} komunikaci možné realizovat pomocí protokolů k tomu určených, které vymezují přenos dat do přesně definovaných časových slotů(Ethernet Powerlink \index{Ethernet Powerlink}, Time-triggered CAN, FlexRay). \index{FlexRay} U webových aplikací žádný takový prvek neexistuje a webová řídicí aplikace se tak stává limitujícím prvkem celé sítě. Existují však metody, které se real-time komunikaci resp. rychlé komunikaci, jak je real-time u webových aplikací všeobecně chápán, mohou velmi přiblížit. V roce 2011 bylo vydáno RFC 6455 \cite{rfc6455}, které zastřešuje nový protokol \index{Websocket} websocket, který umožňuje propojení serveru a klientské části aplikace socketem a je tak možné přenášet informace velmi vysokou rychlostí, což doposud nebylo prakticky téměř možné realizovat.

V následující části práce bude rozebrána problematika komunikace senzorické sítě s webovou řídicí aplikací, ze které vyplyne, že nejvhodnějším řešením je naprogramovat jednotlivé členy senzorické sítě co nejvíce ní\-zko\-ú\-rov\-ňo\-vě, následně je propojit s řídicím serverem, na kterém poběží \index{Node.js} Node.js real-time server (asynchronní single-thread) pro zpracovávání požadavků a zároveň zde poběží server pro webovou aplikaci, která bude využívat websocket protokolu coby nástroje pro komunikaci s tímto serverem. Zároveň je tato senzorická síť uváděna na příkladu administrativní budovy resp. jakéhokoliv objektu kde se běžně pohybují lidé a využívají konvenční elektroinstalaci, tzn. například domácí objekty, popřípadě jiné objekty podobného charakteru kde má využití této sítě praktický přínos.
\chapter{Real-time komunikace}
Real-time komunikace představuje významný prvek v aplikacích, kde je zapotřebí velmi rychlých reakcí systému. Zpravidla se za real-time aplikaci považuje systém, který řeší časové korekce posílaných signálů a tedy vzájemnou časovou synchronizaci vysílače a přijímače. Obecně lze však za real-time aplikaci uvažovat systém, který reaguje na požadavky bez zbytečného dopravního zpoždění, které je například u webových aplikací naprosto běžné. Předejít však dopravnímu zpoždění u webových aplikací není možné. Důvod je prostý. Webová aplikace musí být dostupná pro všechny uživatele na celém světě a z toho plyne, že každý uživatel je na jiném geografickém místě a čas potřebný k dostání informace ke koncovým uživatelům není stejný. Tento problém lze částečně vyřešit distribuovaným systémem, kdy se servery přibližují uživatelům, což prakticky dělají například streamovací portály jako je YouTube. Toto řešení má svá omezení a proto druhým způsobem, jak ušetřit čas při komunikaci s koncovým prvkem, je zjednodušit komunikační protokol, nebo se omezit na co nejméně zbytečné režie a to i za tu cenu, že nedojde ke stoprocentnímu přenosu informace.

\section{Hardwarové prostředky senzorické sítě}
Hardwarové prostředky této sítě nejsou v současné chvíli nijak přesně definovány. Je tedy možné síť navrhnout libovolným způsobem. Vzhledem ke komplikovanosti celé problematiky bude tato síť striktně metalická paketová. Taková síť se tedy skládá v nejmenší konfiguraci pouze z koncového členu a serveru. S narůstajícím počtem koncových členů je zapotřebí síť patřičně rozšiřovat. Výhodou tohoto systému je fakt, že se daná síť nijak neliší od běžných metalických ethernetových sítí, tzn. že lze využít veškeré dostupné prostředky pro tvorbu této sítě a není zapotřebí vyvíjet zbytečně drahá nová zařízení.

Celá síť se tak skládá z klasického ethernetového vedení a rozbočovačů, přepínačů popř. směrovačů. Zbývá tedy vyřešit server a koncové členy. Zde však záleží na praktické aplikaci. Vezmeme-li však v úvahu nejobyčejnější systém, server pak může být prakticky jakýkoliv počítač, který dokáže zpracovat příchozí požadavky. Tzn. musí být dostatečně výkonný a pro lepší bezpečnost celého systému také redundantní (nebo alespoň některé kritické komponenty v něm). Redundanci komponent však dobře řeší klasické servery, kde jsou redundantní například zdroj, pevné disky, řadiče a dále duální paměti popř. procesory.

Samotné koncové prvky se pak sestávají z nízkoodběrových procesorů, které mají menší, pro danou aplikaci však dostatečný výkon. Zde opět záleží na daném účelu koncového zařízení. Pokud má sloužit jako koncentrátor, tedy zařízení sbírající data ze senzorů, potřebuje větší výkon než například termální čidlo. Výkon koncového prvku je tak dán samotným programem, který na tomto prvku poběží.

Tato síť je tedy v takovém stavu, kdy je zapojen server (nejlépe na nezávislém napájení) a senzory jsou zapojeny v ethernetové síti pomocí běžných síťových prvků. Důležité je však vyřešit co se stane, když vypadne napájení? V tomto okamžiku síť prakticky přestane fungovat. Toto se nijak neliší od např. běžné zapojení elektroinstalace. Sice by šlo zajistit napájení koncových prvků, protože server může být zapojen na více nezávislých zdrojích elektrické energie, to však nebude např. v rodinném domě běžné. Horší případ nastane, když vypadne připojení k internetu. Zde by se nejednalo o problém, pokud by se server nacházel v řízeném objektu. Jediný efekt by byl ten, že by nebylo možné server ovládat vzdáleně. Horší situace ovšem nastane v okamžiku, kdy je server umístěn ve vzdálené serverovně. V takovém případě je pro tuto senzorickou síť potřeba vyřešit tzv. disaster solution, tedy nějaký fallback zařízení při selhání. Samotné koncové členy musí vědět jak se chovat bez příchozího signálu. To většinou není problém, protože paradoxně není většinou potřeba řešit jejich chování. To je nutné pouze v případě zabezpečení objektů. Starostí koncových členů totiž není např. vypnout světlo, pokud není přítomen signál. V takovém objektu je však zapotřebí zařadit do sítě zařízení, které bude přijímat od serveru povely a obsluhovat síť. V případě přerušení spojení se serverem převezme toto zařízení kontrolu nad sítí a uvede objekt do dočasného módu, než se problém vyřeší, nebo než přijede servis. Bude tak možné i nadále ovládat alespoň na základní úrovni většinu zařízení.

\section{Real-time ve webových aplikacích}
Ve webových aplikacích žádný real-time jako takový v podstatě neexistuje. 
%TODO AJAX, websocket

\section{TCP}
Protokol TCP je jedním ze dvou transportních protokolů, které tento systém využívá.
%TODO Co to je, jak to funguje a k čemu je to dobré.
\begin{figure}[h]
    \centering
	\makebox[\textwidth]{\includegraphics[width=\textwidth]{img/tcp.png}}
	\caption{TCP stavový diagram} %TODO zdroj http://commons.wikimedia.org/wiki/File:Tcp_state_diagram_fixed_new.svg
    %\label{fig:awesome_image}
\end{figure}

\section{UDP}
\chapter{Volba vhodné technologie}

\section{Prvky senzorické sítě}

\section{Real-time server}

\section{Databázový server}
\cite{redis-benchmark}

\begin{minted}[linenos,breaklines]{text}
$ redis-benchmark -q -n 100000 -d 256
PING_INLINE: 212314.23 requests per second
PING_BULK: 211416.50 requests per second
SET: 131752.31 requests per second
GET: 199600.80 requests per second
INCR: 213219.61 requests per second
LPUSH: 213219.61 requests per second
LPOP: 204918.03 requests per second
SADD: 214592.28 requests per second
SPOP: 212765.95 requests per second
LPUSH (needed to benchmark LRANGE): 213675.22 requests per second
LRANGE_100 (first 100 elements): 45269.35 requests per second
LRANGE_300 (first 300 elements): 15586.04 requests per second
LRANGE_500 (first 450 elements): 9325.75 requests per second
LRANGE_600 (first 600 elements): 6472.49 requests per second
MSET (10 keys): 131578.95 requests per second
\end{minted}

\begin{minted}[linenos,breaklines]{text}
$ redis-benchmark -q -n 100000 -d 256 -P 16
PING_INLINE: 1612903.25 requests per second
PING_BULK: 2127659.75 requests per second
SET: 1086956.50 requests per second
GET: 1351351.38 requests per second
INCR: 1219512.12 requests per second
LPUSH: 934579.44 requests per second
LPOP: 1030927.81 requests per second
SADD: 1265822.75 requests per second
SPOP: 1562499.88 requests per second
LPUSH (needed to benchmark LRANGE): 990099.00 requests per second
LRANGE_100 (first 100 elements): 35186.49 requests per second
LRANGE_300 (first 300 elements): 8521.52 requests per second
LRANGE_500 (first 450 elements): 5236.70 requests per second
LRANGE_600 (first 600 elements): 3888.48 requests per second
MSET (10 keys): 207468.88 requests per second
\end{minted}

\subsection{RESP protokol}
Redis databáze komunikuje interně přes TCP v RESP (Redis Serialization Protocol) formátu. RESP používá celkem 5 typů dat. Vždy platí, že první byte je byte určující o jaký formát se jedná:

\begin{itemize}
\itemsep0em
\item \texttt{+} jednoduchý string
\item \texttt{-} error
\item \texttt{:} integer
\item \texttt{\$} bulk string (binary safe)
\item \texttt{*} array
\end{itemize}

Následuje samotný obsah, nebo dodatečné informace, například o délce a vše je ukončeno pomocí \texttt{CRLF} (\texttt{\textbackslash r\textbackslash n}). Postupně tedy přenášené informace mohou vypadat například takto:

\begin{itemize}
\itemsep0em
\item \texttt{+PONG\textbackslash r\textbackslash n}
\item \texttt{-Error 123\textbackslash r\textbackslash n}
\item \texttt{:54986\textbackslash r\textbackslash n}
\item \texttt{\$4\textbackslash r\textbackslash nPING\textbackslash r\textbackslash n} (první část určuje délku bulk stringu, NULL je pak \texttt{\$-\textbackslash r\textbackslash n})
\item \texttt{*2\textbackslash r\textbackslash n\$3\textbackslash r\textbackslash nGET\textbackslash r\textbackslash n\$3\textbackslash r\textbackslash nkey\textbackslash r\textbackslash n} (první je délka pole, následuje kombinace předchozích)
\end{itemize}

Redis server potom přijímá podle bulk stringů obsahující jednotlivé instrukce. Tento protokol je velmi důležitý, protože i koncentrátory posílají data (přes TCP i UDP) v RESP formátu, je tak možné data posílat přímo do databáze. Tato vlastnost však není využívána, protože je vhodné, aby byl jako prostředník server a například zjišťoval aktivitu koncentrátorů. Každopádně tato možnost zde je a pokud by bylo zapotřebí ukládat data tou nejrychlejší cestou, přímý přístup do databáze je tímto možný a funkční.

\section{Webová aplikace}
\chapter{Program obsluhující koncentrátory}

\chapter{Komunikační protokol}

\chapter{Real-time server}

\chapter{Databázový server}

\chapter{Webová aplikace}
\chapter{Praktická aplikace}

\chapter{Rozšíření stávajícího řešení}
IPv6, Bezdrátová komunikace, Zabezpečení, Další prvky sítě
\chapter{Závěr}

%TODO říct, že je možné metalická i bezdrátová síť



\begin{thebibliography}{99}
\addcontentsline{toc}{chapter}{Literatura}
%http://www.citace.com/

\bibitem{rfc6455}
FETTE, Ian a Alexey MELNIKOV. The WebSocket Protocol. \textit{The Internet Engineering Task Force} [online]. 2011 [cit. 2015-05-20]. Dostupné z: \url{https://tools.ietf.org/html/rfc6455}

\bibitem{real-time}
LAMMERMANN, Sebastian. Ethernet as a Real-Time Technology. \textit{Lammermann.eu} [online]. 2008 [cit. 2015-05-20]. Dostupné z: \url{http://www.lammermann.eu/wb/media/documents/real-time_ethernet.pdf}

\bibitem{mistrovstvi}
SOSINSKY, Barrie A. \textit{Mistrovství – počítačové sítě}. Vyd.~1. Brno: Computer Press, 2011, 840~s. Mistrovství (Computer Press). ISBN 978-80-251-3363-7. 

\bibitem{cube}
\textit{Getting started with STM32CubeF2 firmware package for STM32F2xx series: User manual} [online]. 2014 [cit. 2015-05-20]. Dostupné z: \url{http://www.st.com/st-web-ui/static/active/en/resource/technical/document/user_manual/DM00111485.pdf}

\bibitem{nodejs}
\textit{Node.js} [online]. [cit. 2015-05-20]. Dostupné z: \url{http://nodejs.org/}

\bibitem{redis}
\textit{Redis.io} [online]. [cit. 2015-05-20]. Dostupné z: \url{http://redis.io/}

\bibitem{redis-cluster}
Redis cluster tutorial. \textit{Redis.io} [online]. 2015 [cit. 2015-05-20]. Dostupné z: \url{http://redis.io/topics/cluster-tutorial}

\bibitem{redis-benchmark}
How fast is Redis?. \textit{Redis.io} [online]. [cit. 2015-05-20]. Dostupné z: \url{http://redis.io/topics/benchmarks}

\bibitem{hal}
\textit{Description of STM32F4xx HAL drivers: User manual} [online]. 2015 [cit. 2015-05-20]. Dostupné z: \url{http://www.st.com/st-web-ui/static/active/en/resource/technical/document/user_manual/DM00105879.pdf}

\bibitem{manual}
\textit{Reference manual - STM32F405xx/07xx, STM32F415xx/17xx, STM32F42xxx and STM32F43xxx advanced ARM\textregistered-based 32-bit MCUs} [online]. 2015 [cit. 2015-05-20]. Dostupné z: \url{http://www.st.com/web/en/resource/technical/document/reference_manual/DM00031020.pdf}

\bibitem{icmp}
POSTEL, Jon. Internet Control Message Protocol. \textit{The Internet Engineering Task Force} [online]. 1981 [cit. 2015-05-27]. Dostupné z: \url{http://tools.ietf.org/html/rfc792}

\bibitem{sails}
\textit{Sails.js} [online]. [cit. 2015-05-20]. Dostupné z: \url{http://sailsjs.org/}

\bibitem{ripe}
Understanding IP Addressing. \textit{RIPE NCC} [online]. [cit. 2015-05-20]. Dostupné z: \url{http://bit.ly/1JvOvaO}

\bibitem{openwrt}
Linux distribution for embedded devices. \textit{OpenWrt} [online]. [cit. 2015-05-20]. Dostupné z: \url{https://openwrt.org/}

\bibitem{tessel}
\textit{Tessel 2} [online]. [cit. 2015-05-20]. Dostupné z: \url{https://tessel.io/}

\bibitem{cluster}
Node.js v0.12.3 Manual \& Documentation. \textit{Cluster} [online]. [cit. 2015-05-20]. Dostupné z: \url{https://nodejs.org/api/cluster.html}

\bibitem{v8}
\textit{V8 JavaScript Engine} [online]. [cit. 2015-05-20]. Dostupné z: \url{https://code.google.com/p/v8/}

\bibitem{heroku}
\textit{Heroku} [online]. [cit. 2015-05-20]. Dostupné z: \url{https://www.heroku.com/}

\bibitem{forever}
A~simple CLI tool for ensuring that a given script runs continuously. \textit{Forever} [online]. [cit. 2015-05-20]. Dostupné z: \url{https://github.com/foreverjs/forever}

\bibitem{q}
A~tool for creating and composing asynchronous promises in JavaScript. \textit{Q} [online]. [cit. 2015-05-20]. Dostupné z: \url{http://documentup.com/kriskowal/q/}

\bibitem{socket}
\textit{Socket.IO} [online]. [cit. 2015-05-20]. Dostupné z: \url{http://socket.io/}

\bibitem{cordova}
Platform for building native mobile applications using HTML, CSS and JavaScript. \textit{Apache Cordova} [online]. [cit. 2015-05-20]. Dostupné z: \url{http://cordova.apache.org/}

\bibitem{meteor}
Running your app on Android or iOS. \textit{Meteor} [online]. [cit. 2015-05-20]. Dostupné z: \url{https://www.meteor.com/try/7}

\bibitem{lifi}
\textit{pureLiFi} [online]. [cit. 2015-05-20]. Dostupné z: \url{http://purelifi.com/}

\bibitem{ternar}
VRÁNA, Jakub. \textit{1001 tipů a triků pro PHP}. Vyd.~1. Brno: Computer Press, 2010, 136 Co je ternární operátor. ISBN 978-80-251-2940-1.

\bibitem{ted}
prof. HAAS, Harald. Wireless data from every light bulb. \textit{TED - Ideas worth spreading} [online]. 2011 [cit. 2015-05-20]. Dostupné z: \url{http://www.ted.com/talks/harald_haas_wireless_data_from_every_light_bulb}

\end{thebibliography}

\addcontentsline{toc}{chapter}{Rejstřík}
\printindex

\end{document}