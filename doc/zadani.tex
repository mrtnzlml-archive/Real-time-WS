\pagestyle{empty}
\begin{titlepage}
\begin{center}
{\Large ZÁPADOČESKÁ UNIVERZITA V PLZNI} \\
\textbf{Fakulta elektrotechnická} \\
Akademický rok \textbf{2014/2015} \\
\vspace{2cm}
{\Huge\sc Zadání bakalářské práce} \\
\end{center}
\vspace{2cm}
\begin{tabular}{ll}
\textit{Jméno a příjmení:} & Martin Zlámal \\ 
\textit{Osobní číslo:} & • \\ 
\textit{Studijní program:} & • \\ 
\textit{Studijní obor:} & • \\ 
\textit{Název tématu:} & • \\ 
\textit{Zadávající katedra:} & • \\ 
\end{tabular} 

\vspace{2cm}
\begin{center}
{\large ZÁSADY PRO VYPRACOVÁNÍ} \\
\end{center}
\begin{enumerate}
\item Prostudujte si a teoreticky zpracujte dostupné materiály k problematice real-time komunikací pro senzorické sítě s přihlédnutím k webovým aplikacím. Seznamte se s hardwarovými prostředky senzorické sítě.
\item Na základě předchozího bodu zvolte vhodnou technologii a navrhněte strukturu programového řešení v závislosti na dostupném hardware.
\item Napište program obsluhující server senzorické sítě s přihlédnutím na real-time komunikaci s koncentračními jednotkami.
\item Napište program obsluhující koncentrační jednotky v uzlech senzorické sítě s přihlédnutím na real-time komunikaci se serverem.
\item Naprogramujte webovou aplikaci umožňující vizualizaci a ovládání senzorické sítě.
\item Rozeberte možnosti praktické aplikace této sítě a její možnosti rozšíření.
\end{enumerate}
\end{titlepage}